\begin{slide}
\pagestyle{headings}
\sf
\header{Practical work: gain from weighted average}
A radioactive source is completely surrounded by two
hemispherical counters, one with 100\% efficiency and the other
with 10\% efficiency only. 
%
An experimenter observes within one minutes $100\pm 10$ decays
in the first counter and $9\pm 3$ in the second.
%

\noindent
Estimate the total decay rate of the source per minute
using the following two methods - compare the results and
their precision:\\[2mm]
{\bfseries  A) unweighted average:}
\begin{enumerate}
\item 
Correct for each counter the observed number for the counter inefficiency
\item
Sum the corrected numbers in both counters to the total
number of decays and determine an uncertainty using error propagation
\end{enumerate}
\vspace{2mm}
{\bfseries  B)  weighted average:}
\begin{enumerate}
\item
Assume that without any inefficiency each of the counters
has 50\% geometrical acceptance, i.e. would see half of the
decays. Determine two separate measurements of the total
decay rate by correcting for each counter the efficiency corrected
rate additionally by the geometric acceptance.
\item
Calculate the weighted mean of the two separate measurements
and an error
\end{enumerate}
%
\end{slide}

