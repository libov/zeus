\begin{slide}
\pagestyle{headings}
\sf
\bheader{  {\normalsize \bf \darkgreen Mini-exercise}
Gain from weighted average}
%
\large
%
Die totale Teilchenrate aus einer Quelle soll gemessen werden.
Um die Quelle herum ist ein hermetischer Detektor gebaut.
Die eine H\"alfte des Detektors misst 
\[N_1 = 100 \pm 10\]
Die andere H\"alfte des Detektors ist sehr ineffizient
und misst (effizienzkorrigiert!) 
\[N_2 = 100 \pm 100\]
%
\vspace{2mm}
{\em  \darkgreen \bfseries
Sch\"atzen Sie die Gesamtrate $N$ mit den folgenden zwei Methoden:}
\begin{enumerate}
\item
$ \hat{N} = N_1 + N_2 $
\item
Multipliziere $N_1$ und $N_2$ jeweils mit Faktor 2 
(beide H\"alften 'sehen' ja 50\%!) 
$\rightarrow$ 
separate Messungen:
\begin{itemize}
\item Erste Messung: 
$ N = 200 \pm 20 $
\item Zweite Messung
$ N = 200 \pm 200 $
\end{itemize}
$\rightarrow$ Bilde das gewichtete Mittel beider Messungen
als Sch\"atzwert $\hat{N}$.
\end{enumerate}
%
\vspace{2mm}

\begin{itemize}
\item[$\rightarrow$]
{\em \darkgreen \bfseries Bestimmen Sie f\"ur beide Verfahren den Fehler auf
$\hat{N}$ (Formeln s. Vorlesung!)}
\end{itemize} 
\end{slide}


