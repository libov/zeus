\begin{slide}
\pagestyle{headings}
\small
\sf 
\header{Overview of Linear least square fit section}
%\renewcommand{\arraystretch}{5.5}
\begin{table}
\setlength{\tabcolsep}{6pt}
\hspace*{-5mm}
\scriptsize
\begin{tabular}{|l|l|l|}
\hline
Part I & Part II & Part III \\
\hline
\begin{minipage}{3.6cm}
\vspace*{1mm}
%\begin{itemize}
\begin{list}{\labelitemi}{\leftmargin=1em}
\item
Reminder of $\chi^2$-fit method 
\item
Linear $\chi^2$-fit examples (Constant, straight
line, parabola, etc.)
\item
{\bfseries \blue Fit of a constant (averaging measurements)}
\item 
One single measurement:
$\chi^2_{min}$ and $\chi^2_{min}+1$,
Hesse matrix
\item
{\magenta Exercise: Two measurements:
perform fit by adding $\chi^2$-parabolas}
\item
Averaging many measurements, results
\item
 {\magenta Exercise: Compare
weighted vs unweighted average}
\end{list}
%\end{itemize}
\end{minipage}
&
\begin{minipage}{5.5cm}
\vspace*{1mm}
\begin{list}{\labelitemi}{\leftmargin=1em}
%\begin{itemize}
\item
{\blue \bfseries $\chi^2$-fit-quality test:}
Example: $\chi^2$ of two measurements
and known true value
\item
$\chi^2$-function for $n$ degrees of freedom
and $\chi^2$-fit probability
%\item
{\scriptsize \red Exercise: plot and study features
of the $\chi^2$-function vs $n$ using the parameterised function}
{\red New: Generate 1000 random experiments with n degrees of
freedom and obtain $\chi^2$ and $\chi^2$-fit probability distributions} 
%\item
%
%\item
%{\red exercise: 
%plot and study features
%of the $\chi^2$-fit-probability}
\item
$\chi^2$ for two measurements with unknown true value
\item
{\red \scriptsize New exercise: Track position measurement in test beam
using 10 detector layers, in each detector 99\%
chance for signal hit and 1\% for random noise hit
$\rightarrow$  Generate 1000 tracks and corresponding
hits and obtain $\chi^2$, $\chi^2$-fit probability 
and measured parameter distributions. 
Try to reject outliers: Method 1: reject track fits with
small $\chi^2$-fit probability, Method 2: iterative,
repeat track-fit and downweight outliers}
\item
{\red Exercise: {\blue \bfseries Outlier rejection}, case
world average of $m_W$, {\scriptsize  study how
the rejection of certain measurements 
change the average and the $\chi^2$-fit probability}}
\item
%{\scriptsize \darkgreen Averaging data with unknown errors}
%\item
{\red New exercise: Upscaling of errors a la PDG to obtain reasonable $\chi^2$}
\item
{\red New exercise: Pulls of single measurements to 
the average}
%\end{itemize}
\end{list}
\end{minipage}
& 
\begin{minipage}{5.cm}
\vspace*{1mm}
\begin{list}{\labelitemi}{\leftmargin=1em}
%\begin{itemize}
\item
General form of linear $\chi^2$
\item
Solution by normal equations
%\item
%General features:
%(Consistency, Unbiasedness, efficiency)
\item
Normal equation solution for 
{\blue \bfseries straight line
fit}
\item
{\magenta Exercise: Learn qualitative features
of straight line fits, e.g. importance of lever arm}
\item
{\red Exercise: Straight line fit and 
detailed error analysis (error ellipse, trajectory
error band)}
\item
{\red New exercise: Coordinate transformation such that the
coordinate center is in the middle of the points $\rightarrow$
study the effect on the parameter errors and correlation}
\item 
{\red New exercise: Add a very precise point at the origin of
the track such that the $p_0$ parameter is basically fixed.
Repeat the track-fit and study the effect on the slope and error}
\item
{\red Exercise: {\blue \bfseries Parabola track fit},
complete analysis: {\it \scriptsize fit, outlier-rejection, 
parameter errors/correlation, trajectory uncertainty,
{\blue \bfseries momentum calculation}}}
%\item
%{\red Exercise: {\blue \bfseries Guessing the right fit function
%for smooth data (polynomial fit of background}}
%\end{itemize}
\end{list}
\vspace*{1mm}
\end{minipage}
\\
\hline
\end{tabular}
\end{table}
\end{slide}
